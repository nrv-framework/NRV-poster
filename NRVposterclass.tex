\documentclass{NRVposter}
\usepackage{graphicx}

\title{The \texttt{NRVposter} class}
\author{T. Couppey, L. Regnacq, }
\institute{Laboratory IMS - UMR CNRS 5218}


\begin{document}
\maketitle

% Tip: for long titles, use \parbox{\linewidth}{text}

\block{Abstract}{This is a description of the class to make posters according 
to NRV's official template.}

\begin{columns}
\column{0.5}
\block{NRV Framework}{
This class is meant to prepare posters for academic conferences, and is 
implemented as a reconfiguration of the \texttt{tikzposter} class.
Previous versions derived from \texttt{sciposter}, but it was not practical to 
emulate the official NRV template from that class when the template finally 
came out.

\innerblock{Requirements and Compilation}
{The NRV corporate colors are taken from the \texttt{NRVcolor} package,
which is therefore necessary to have together with this class.
The class uses Carlito, a free clone of Calibri, so \texttt{NRVposter} can 
be compiled with any \LaTeX\ compiler.}

\innerblock{Class Options and Caveats}
{The main options taken by the class are \texttt{portrait} (default) and 
\texttt{landscape}. Please note that the class is still experimental and weird 
things may happen, especially if you try to compile for sizes different from A0.

Note especially that the \textcolor{red}{title \emph{will not wrap}} if it is 
too long; this is actually a \texttt{tikzposter} problem.
In case of a long title that needs to run over multiple lines, the trick is 
simple: write \texttt{\textbackslash title\{\textbackslash 
parbox\{\textbackslash linewidth\}\{Your very long title\}\}}.


This is a first version and still somewhat experimental.
Let me know if you find any glaring errors or things you want implemented.
}
}

\block{Compilation \emph{will} fail on first attempt with hyperref}{
Due to a bug in \texttt{tikzposter}, compilation will always fail in the 
absence of an \texttt{.aux} file, i.e. when compiling for the first time, in 
case you included the \texttt{hyperref} package.
Just compile again and things will work fine.
This will eventually be corrected in \texttt{tikzposter}.
}

\column{0.5}
\block{Writing}
{To write sections of text, use the \texttt{\textbackslash 
block\{Heading\}\{Text\}} command; inside these blocks, subsections can be 
added with \texttt{\textbackslash innerblock\{Heading\}\{Text\}}.

Blocks will by default take the full width of the page if you have not split it 
into columns with the \texttt{columns} environment.
Add a \texttt{\textbackslash column\{x\}} command to start writing in a new 
column, where \texttt{x} is a number between 0 and 1 whose function you should 
be able to guess.

No surprises for adding images, which are probably anyway the thing you should 
mostly put in a poster.
}

\begingroup
\colorlet{framecolor}{NRVcyan}
\block{Outline}
{If you want to have a cyan outline for your Conclusion box, as in the 
template, you can set the \texttt{framecolor} variable \emph{before} opening 
the block:
\begin{quote}\texttt{\noindent
\textbackslash begingroup\\
\textbackslash colorlet\{framecolor\}\{NRVcyan\}\\
\textbackslash block\{Outline\}\{blah blah\}\\
\textbackslash endgroup
}\end{quote}
Use \texttt{\textbackslash begingroup} and \texttt{\textbackslash endgroup} to 
make the change local and not permanently change the outline colour.
}
\endgroup

\block{Suggestions \& Links}{
Refer to \texttt{tikzposter}'s manual for how to customise the poster layout.

To insert figures with text wrapped around them, use the \texttt{wrapfig} 
package.

For best results in plots, the \texttt{pgfplots} package makes it easy to keep 
the same font style and size in plots as in the text.
}

\end{columns}


\end{document}

